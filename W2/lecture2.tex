\documentclass{amsart}

\usepackage{todonotes}
\usepackage{bm}

\def\Z{\mathbb{Z}}
\def\C{\mathbb{C}}
\def\F{\mathbb{F}}

\newcommand{\boxtimes}{\odot}
\newcommand{\stacks}{\Delta}
\newcommand{\toposX}{\mathfrak X}
\newcommand{\etale}{\'{e}tale\,}
\newcommand{\Cech}{\v{C}ech\,}
\newcommand{\bigdot}{\bullet}
\newcommand{\coproduct}{\coprod}
\newcommand{\catname}[1]{{\sffamily\upshape{{#1}}}}
\newcommand{\sset}{\catname{sSet}}
\newcommand{\topp}{\catname{Top}}
\DeclareMathOperator{\Sing}{Sing}
\DeclareMathOperator{\Sub}{\mathcal P}
\DeclareMathOperator{\coeq}{\mathfrak CE}
\DeclareMathOperator{\group}{\mathfrak GPD}
\DeclareMathOperator{\colim}{colim}
\DeclareMathOperator{\Top}{\mathsf{Top}}
%\DeclareMathOperator{\topp}{Top}
%\DeclareMathOperator{\sSet}{sSet}
\DeclareMathOperator{\PTop}{Fun^{\text{pres}}}
\DeclareMathOperator{\holim}{holim}
\DeclareMathOperator{\cosk}{cosk}
\DeclareMathOperator{\hocolim}{hocolim}
\DeclareMathOperator{\bd}{\partial}
\DeclareMathOperator{\U}{\mathcal{U}}
\DeclareMathOperator{\calU}{\mathcal{U}}
\DeclareMathOperator{\calW}{\mathcal{W}}
\DeclareMathOperator{\calE}{\mathcal{E}}
\DeclareMathOperator{\calB}{\mathcal{B}}
\DeclareMathOperator{\calK}{\mathcal{K}}
\DeclareMathOperator{\Sp}{Sp}
\DeclareMathOperator{\Simp}{\bold{CX}}
\DeclareMathOperator{\calF}{\mathcal{F}}
\DeclareMathOperator{\calG}{\mathcal{G}}
\DeclareMathOperator{\Hom}{Hom} \DeclareMathOperator{\bP}{\beta P}
\DeclareMathOperator{\HH}{H} \DeclareMathOperator{\BU}{BU}
\DeclareMathOperator{\id}{id} \DeclareMathOperator{\Fun}{Fun}
\DeclareMathOperator{\calC}{\mathcal{C}}
\DeclareMathOperator{\calI}{\mathcal{I}}
\DeclareMathOperator{\calJ}{\mathcal{J}}
\DeclareMathOperator{\SSet}{\mathcal{S}}
\DeclareMathOperator{\calS}{\mathfrak{S}}
\DeclareMathOperator{\calX}{\mathcal{X}}
\DeclareMathOperator{\red}{\text{hyp}}
\DeclareMathOperator{\calY}{\mathcal{Y}}
\DeclareMathOperator{\op}{op}
\DeclareMathOperator{\calD}{\mathcal{D}}
\DeclareMathOperator{\Ind}{Ind} \DeclareMathOperator{\Acc}{Acc}
\DeclareMathOperator{\Pre}{\PTop}
\DeclareMathOperator{\calP}{\mathcal{P}} \topmargin=0in
\oddsidemargin=0in \evensidemargin=0in \textwidth=6.5in
\textheight=8.5in

\newtheorem{theorem}{Theorem}[subsection]
\newtheorem{lemma}[theorem]{Lemma}
\newtheorem{proposition}[theorem]{Proposition}
\newtheorem{corollary}[theorem]{Corollary}
\newtheorem{fact}[theorem]{Fact}
% \newtheorem{bigtheorem}{Theorem}

\usepackage{tikz-cd}

\theoremstyle{definition}
\newtheorem{definition}[theorem]{Definition}
\newtheorem{example}[theorem]{Example}
\newtheorem{counterexample}[theorem]{Counterexample}
\newtheorem{conjecture}[theorem]{Conjecture}
\newtheorem{postulate}[theorem]{Postulate}
\newtheorem{remark}[theorem]{Remark}

%\newtheorem{notation}[theorem]{Notation}
%\newtheorem{question}[theorem]{Question}
%\newtheorem{variant}[theorem]{Variant}


\begin{document}

\title{Lecture 2 -- Simplicial Sets}
\author{Roger Murray}
\date{\today}

\maketitle
\section{Notation}
Categories will typically be sans serif (\textit{i.e.} \topp, \sset, etc.).
If $\calC$ is an arbitrary category then we denote the the set of $\calC$-morphisms from
objects $A,B\in \calC$ by $\mathcal{C}(A,B)$.

\section{Garbage}

Recall the definition of the \textit{standard topological $n$-simplex} as the set
$$
\Delta^n_{\Top} := \left\{ (t_0,\cdots, t_n)\in \mathbb{R}^{n+1} \, \Big| \,
  \sum t_i = 1 \text{ and } t_i\geq 0 \text{ for all }i \right\}
$$
An alternative way of thinking of $\Delta^n_{\Top}$ is as the convex hull of
vertices $v_i=(0,\ldots,1,\ldots,0)$.
We then have maps codegeneracy maps $s^i:\Delta^{n+1}_{\Top} \to \Delta^n_{\Top}$,
and coface maps $d^i: \Delta^{n-1}_{\Top} \to \Delta^n_{\Top}$ defined by
\begin{align*}
  & s^i(t_0,\ldots,t_{n+1}) = (t_0,\ldots,t_i+t_{i+1},\ldots,t_{n+1}) \\
  & d^i(t_0,\ldots,t_{n-1}) = (t_0,\ldots,t_i,0,t_{i+1},\ldots,t_n)
\end{align*}
Clearly $d^i$ is just the map embedding $\Delta^{n-1}_{\Top}$ as the $i^{th}$ face of $\Delta^{n}_{\Top}$.
Similarly $s^i$ is a retration of $\Delta^{n+1}_{\Top}$ minus the $i^{th}$ vertice $v_i$ onto
the face opposite $v_i$.
\\
\\
\indent
Given any topological space $X\in\Top$ we define the \textit{singular $n$-simplices of $X$} to be the maps
$$
\Sing(X)_n := \Top( \Delta^n_{\Top},\, X)
$$
This turns out to be an important example of something called a simplicial object.
Before we can define what a simplicial object is, we must first define the
\textit{simplex category} $\Delta$. The objects of $\Delta$ are the ordered sets
$[n]=\{0,1,\ldots,n\}$, and the morphisms $f:[m]\to[n]$ are the
weakly-order-preserving (\textit{i.e.} non-decreasing) functions. Similarly to
above we have maps $s^i:[n+1]\to [n]$ and $d^i:[n-1]\to [n]$ given by
repeating the $i$, and skipping $i$ respectively. The following lemma says
that these maps are the only maps we care about.
\begin{lemma}
  Any map $f$ in $\Delta$ is the composition of a series
  of $d^i$ and $s^j$.
\end{lemma}
\begin{proof}[Proof sketch]
  If you have a map $f:[m]\to[n]$ then you have the inequality $f(0)\leq f(1) \leq \cdots \leq f(m)$.
  We get unique elements of the form $g_0<\cdots < g_k$ for $k\leq \min \{m,n\}$. By composing
  $d^0$ with itself $g_0$ times we get a map that sends $[m]$ to $[g_0,g_0+1,\ldots,g_0+m]$.
  We repeat $g_0$ with $s^0$ as many times as it occurs in the sequence of $f(i)$s. We then simply
  repeat this process inductively on the $g_i$ for $0 < i\leq k$.
\end{proof}


\begin{definition}
  A \textit{simplicial object} is a functor $X: \Delta^{\op} \to \mathsf{Set}$.
  %\quad \quad\quad \quad \quad \quad \quad \,
  Similarly a \textit{cosimplicial object} is a functor $X:\Delta \to \mathsf{Set}$.
\end{definition}


%%%%%%%%%%%%%%%%%%%%%%%%%%%%%%%%%%
%%%%%%%%%%%%%%%%%%%%%%%%%%%%%%%%%%
%%%%%%%%%%%%%%%%%%%%%%%%%%%%%%%%%%
%%%%%%%%%%%%%%%%%%%%%%%%%%%%%%%%%%
% Bibliography %%%%%%%%%%%%%%%%%%%
%%%%%%%%%%%%%%%%%%%%%%%%%%%%%%%%%%
\begin{thebibliography}{10}
asdasd
\end{thebibliography}


% \begin{thebibliography}{99}

% \bibitem{adamek} Ad\'{a}mek and Rosicky. {\it Locally Presentable
% and Acessible Categories}. Cambridge University Press, Cambridge,
% 1994.

% \end{thebibliography}

\end{document}
