\documentclass{amsart}
\usepackage{todonotes}
\usepackage{amssymb}
\usepackage{mathabx}

\def\Z{\mathbb{Z}}
\def\C{\mathbb{C}}
\def\F{\mathbb{F}}

%\newcommand{\boxtimes}{\odot}
\newcommand{\stacks}{\Delta}
\newcommand{\toposX}{\mathfrak X}
\newcommand{\etale}{\'{e}tale\,}
\newcommand{\Cech}{\v{C}ech\,}
\newcommand{\bigdot}{\bullet}
\newcommand{\coproduct}{\coprod}
\newcommand{\catname}[1]{{\sffamily\upshape{{#1}}}}
\newcommand{\sset}{\catname{sSet}}
\newcommand{\topp}{\catname{Top}}
\DeclareMathOperator{\Sing}{Sing}
\DeclareMathOperator{\Sub}{\mathcal P}
\DeclareMathOperator{\coeq}{\mathfrak CE}
\DeclareMathOperator{\group}{\mathfrak GPD}
\DeclareMathOperator{\colim}{colim}
\DeclareMathOperator{\Top}{\mathsf{Top}}
\DeclareMathOperator{\Natl}{Natl}
%\DeclareMathOperator{\topp}{Top}
\DeclareMathOperator{\msset}{\mathsf{sSet}}
\DeclareMathOperator{\PTop}{Fun^{\text{pres}}}
\DeclareMathOperator{\holim}{holim}
\DeclareMathOperator{\cosk}{cosk}
\DeclareMathOperator{\hocolim}{hocolim}
\DeclareMathOperator{\bd}{\partial}
\DeclareMathOperator{\U}{\mathcal{U}}
\DeclareMathOperator{\calU}{\mathcal{U}}
\DeclareMathOperator{\calW}{\mathcal{W}}
\DeclareMathOperator{\calE}{\mathcal{E}}
\DeclareMathOperator{\calB}{\mathcal{B}}
\DeclareMathOperator{\calK}{\mathcal{K}}
\DeclareMathOperator{\Sp}{Sp}
\DeclareMathOperator{\Simp}{\bold{CX}}
\DeclareMathOperator{\calF}{\mathcal{F}}
\DeclareMathOperator{\calG}{\mathcal{G}}
\DeclareMathOperator{\Hom}{Hom} \DeclareMathOperator{\bP}{\beta P}
\DeclareMathOperator{\HH}{H} \DeclareMathOperator{\BU}{BU}
\DeclareMathOperator{\id}{id} \DeclareMathOperator{\Fun}{Fun}
\DeclareMathOperator{\calC}{\mathcal{C}}
\DeclareMathOperator{\calI}{\mathcal{I}}
\DeclareMathOperator{\calJ}{\mathcal{J}}
\DeclareMathOperator{\SSet}{\mathcal{S}}
\DeclareMathOperator{\calS}{\mathfrak{S}}
\DeclareMathOperator{\calX}{\mathcal{X}}
\DeclareMathOperator{\red}{\text{hyp}}
\DeclareMathOperator{\calY}{\mathcal{Y}}
\DeclareMathOperator{\op}{op}
\DeclareMathOperator{\calD}{\mathcal{D}}
\DeclareMathOperator{\Ind}{Ind} \DeclareMathOperator{\Acc}{Acc}
\DeclareMathOperator{\Pre}{\PTop}
\DeclareMathOperator{\calP}{\mathcal{P}} \topmargin=0in
\oddsidemargin=0in \evensidemargin=0in \textwidth=6.5in
\textheight=8.5in

\newtheorem{theorem}{Theorem}[section]
\newtheorem{lemma}[theorem]{Lemma}
\newtheorem{proposition}[theorem]{Proposition}
\newtheorem{corollary}[theorem]{Corollary}
\newtheorem{fact}[theorem]{Fact}
% \newtheorem{bigtheorem}{Theorem}
\newtheorem{properties}[theorem]{Properties}

\usepackage{tikz-cd}

\theoremstyle{definition}
\newtheorem{definition}[theorem]{Definition}
\newtheorem{example}[theorem]{Example}
\newtheorem{counterexample}[theorem]{Counterexample}
\newtheorem{conjecture}[theorem]{Conjecture}
\newtheorem{postulate}[theorem]{Postulate}
\newtheorem{remark}[theorem]{Remark}

%\newtheorem{notation}[theorem]{Notation}
%\newtheorem{question}[theorem]{Question}
%\newtheorem{variant}[theorem]{Variant}


\begin{document}

\title{Lecture 2 -- Simplicial Sets}
\author{notes anotated by Roger Murray}
\date{\today}

\maketitle
\section{Notation}
Categories will typically be sans serif (\textit{i.e.} \topp, \sset, etc.) or caligraphric
(\textit{i.e.} $\mathcal{C}$, $\mathcal{S}$, etc.).
If $\calC$ is an arbitrary category then we denote the the set of $\calC$-morphisms from
objects $A,B\in \mathcal{C}$ by $\mathcal{C}(A,B)$.

\section{Simplices}

Recall the definition of the \textit{standard topological $n$-simplex} as the set
$$
\Delta^n_{\Top} := \left\{ (t_0,\cdots, t_n)\in \mathbb{R}^{n+1} \, \Big| \,
  \sum t_i = 1 \text{ and } t_i\geq 0 \text{ for all }i \right\}
$$
Alternatively we may think of $\Delta^n_{\Top}$ as the convex hull of
vertices $v_i=(0,\ldots,1,\ldots,0)$.
We then have \textit{codegeneracy maps} $s^i:\Delta^{n+1}_{\Top} \to \Delta^n_{\Top}$,
and \textit{coface maps} $d^i: \Delta^{n-1}_{\Top} \to \Delta^n_{\Top}$ defined by
\begin{align*}
  & s^i(t_0,\ldots,t_{n+1}) = (t_0,\ldots,t_i+t_{i+1},\ldots,t_{n+1}) \\
  & d^i(t_0,\ldots,t_{n-1}) = (t_0,\ldots,t_i,0,t_{i+1},\ldots,t_n)
\end{align*}
Clearly $d^i$ is just the map embedding $\Delta^{n-1}_{\Top}$ as the $i^{th}$ face of $\Delta^{n}_{\Top}$, and $s^i$ is a retration of $\Delta^{n+1}_{\Top}$ minus the $i^{th}$ vertice $v_i$ onto
the face opposite $v_i$.
\\
\\
\indent
Given any topological space $X\in\Top$ we define the \textit{singular $n$-simplices of $X$} to be the set of maps
$$
\Sing(X)_n := \Top( \Delta^n_{\Top},\, X)
$$
This turns out to be an important example of something called a simplicial object.
Before we can define what a simplicial object is, we must first define the
\textit{simplex category} $\Delta$. The objects of $\Delta$ are the ordered sets
$[n]=\{0,1,\ldots,n\}$, and the morphisms $f:[m]\to[n]$ are the
weakly-order-preserving (\textit{i.e.} non-decreasing) functions. Similarly to
above we have maps $s^i:[n+1]\to [n]$ and $d^i:[n-1]\to [n]$ given by
repeating the $i$, and skipping $i$ respectively. The following lemma says
that these maps are the only maps we care about.
\begin{lemma}\label{compos}
  Any map $f$ in $\Delta$ is the composition of some $d^i$ and $s^j$.
\end{lemma}
\begin{proof}[Proof sketch]
  If you have a map $f:[m]\to[n]$ then you have the inequality $f(0)\leq f(1) \leq \cdots \leq f(m)$.
  We obtain a unique chain of elements of the form $g_0<\cdots < g_k$
  for $k\leq \min \{m,n\}$. By composing
  $d^0$ with itself $g_0$ times we get a map that sends $[m]$ to $[g_0,g_0+1,\ldots,g_0+m]$.
  We repeat $g_0$ with $s^0$ as many times as it occurs in the sequence of $f(i)$s. We then simply
  repeat this process inductively on the $g_i$ for $0 < i\leq k$.
\end{proof}


\begin{definition} Let $\mathcal{C}$ be a category. A \textit{simplicial object
  in $\mathcal{C}$} is a functor
  $X: \Delta^{\op} \to \mathcal{C}$.
  %\quad \quad\quad \quad \quad \quad \quad \,
  Similarly a \textit{cosimplicial object in $\mathcal{C}$} is a functor
  $Y:\Delta \to \mathcal{C}$.
\end{definition}
We write $X_n$ and $Y^n$ for $X([n])$ and $Y([n])$ respectively, and hence we
will often use the notation $X_\bullet$ and $Y^\bullet$ for $X$ and
$Y$ respectively.
Similarly we write $d_i,s_i$ and $d^i,s^i$ for the obvious maps in $\mathcal{C}$.


\begin{example}
  The standard simplices $\Delta^\bullet_{\Top}$ is a cosimplicial space.
\end{example}
\begin{example}
  Given any space $X\in \Top$ we therefore have that
  $\Sing(X)_\bullet$ is a simplicial set acting on objects by
  $[n]\mapsto \Top(\Delta^n_{\Top}, \, X)$, and on morphisms by
  $(f:[m]\to [n])\mapsto f_*$ where $f_*$ is precomposition
  with $f$.
\end{example}

\begin{properties}\label{simplprops}
  The degeneracy, and face maps belonging to any simplicial object $X$ in
  $\mathcal{C}$ satisfy the following,
  
  \begin{enumerate}
  \item $d_i\circ d_j = d_{j-1} \circ d_i$ for $i<j$
  \item $s_i\circ s_j = s_{j+1}\circ s_i$ for $i\leq j$
  \item Lastly, 
    \begin{flalign*}
      d_i\circ s_j = 
     \begin{cases}
       s_{j-1}\circ d_i &\text{ if } i<j\\
       \id &\text{ if } i=j,j+1 \\
       s_j\circ d_{i-1} &\text{ if } i>j+1
     \end{cases}
    \end{flalign*}
  \end{enumerate}
\end{properties}
We refer to these properties as the \textit{simplicial properties}.
A dual statement holds for cosimplicial objects. These (dual) properties
are obvious in the case of $\Delta$.
We next construct a functor which is adjoint to $\Sing(\,\text{--}\,)_\bullet$.
\\
\\
\indent
Let $X$ be a simplicial object in a category $\mathcal{C}$.
\begin{definition}
  An $n$-simplex $x\in X_n$ is \textit{degenerate} if there exists
  some $y\in X_{n-1}$ such that $x=s_i(y)$ for some $0\leq i\leq n-1$.
\end{definition}
\begin{definition}
  The \textit{geometric realisation of $X$} is the space given by
  $$
  |X_\bullet| = \coprod_{n\geq 0} X_n\times \Delta^n_{\Top}  \,\Big/\sim
  $$
  where the equivalence relation is given by identifying
  $(f^*x,u)\sim (y,f_*u)$ for any $f\in \Delta([m],[n])$, and
  $x\in X_n$, and $u\in \Delta^m_{\Top}$.
\end{definition}

By \ref{compos} it is enough to consider when $f$ is a degeneracy, or face map.
\\
\\
\indent
We have the following useful lemma,
\begin{lemma}\label{degen}
  Every object $x\in X_n$ may be written uniquely as
  $s_{i_1}\circ s_{i_2} \circ \cdots \circ s_{i_l} (y)$ such that
  $i_1>\cdots > i_n$ and $y\in X_\bullet$ is non-degenerate.
\end{lemma}
\begin{proof}
  By (2) of \ref{simplprops} it is easy to see that we can find an
  increasing sequence $i_j$. Furthermore this sequence must terminate since
  $x\in X_n$ and each degeneracy map increases degree by $1$. Thus we obtain
  a $y$ such that $x=s_{i_1}\circ \cdots \circ s_{i_n} (y)$. All that is
  left to check is that these are unique. This is immediate by starting
  at $x$ and working backwards. 
\end{proof}

Given two simplicial sets $X$ and $Y$ we obtain another simplicial set
$X\times Y$ defined by $(X\times Y)_n = X_n\times Y_n$. The above lemma
implies that we may combine two degenerate simplices $x\in X_n$
and $y\in Y_n$ to get a non-degenerate simplex $(x,y)$. This will be important
below.

\begin{lemma}\label{realisation}
  Given a simplicial set $X$ its geometric
  realisation $|X|$ is a CW-complex with an $n$-cell for every non-degenerate
  $n$-simplex $x\in X_n$.
\end{lemma}


For a classical proof of this see page 56 of \cite{May}.



\begin{proposition}
  The functors
  \begin{tikzcd}
    {|\text{ -- }|: \mathsf{sSet}} \arrow[r, shift right] &
    {\arrow[l, shift right] \Top : \Sing}
  \end{tikzcd}
are adjoint.
\end{proposition}
\begin{proof}
  We wish to show that, given any $X_\bullet\in \mathsf{sSet}$ and
  any $Y\in \Top$, the sets
  $\mathsf{sSet}(X_\bullet, \Sing(Y)_\bullet)$ and
  $\Top(|X_\bullet|, Y)$ are in bijection.
  This proof relies on the fact that any map $f_n\in \mathsf{sSet}(X_n,
  \Sing(Y)_n)$ corresponds to a map
  $\widetilde{f_n}: X_n\times \Delta^n_{\Top}\to Y$ where $\widetilde{f_n}(x,u) = f_n(x)(u)$.
  If we now consider any map $g\in \Delta([m],[n])$ then we get the following commutative
  diagram by definition
  \[
    \begin{tikzcd}
      X_n \arrow[r,"f_n"] \arrow[d,"X(g)"] & {\mathsf{sSet(\Delta^n_{\Top},Y)}}
      \arrow[d, "\left(\Delta^\bullet_{\Top} (g)\right)^*"]\\
      X_m \arrow[r, "f_m"] &  {\mathsf{sSet(\Delta^m_{\Top},Y)}}
    \end{tikzcd}
  \]
  By the existence of $\widetilde{f_n}$ in relation to $f_n$ this gives us the
  following commutative diagram
  \[
    \begin{tikzcd}
      X_n\times \Delta^m_{\Top} \arrow[r,"\id \times \Delta_{\Top}^\bullet(g)"]
      \arrow[d,"X(g)\times \id"]
      & X_n\times \Delta^n_{\Top}
      \arrow[d, "\widetilde{f_n}"]\\
      X_m \times \Delta^m_{\Top} \arrow[r, "\widetilde{f_m}"]
      &  {\mathsf{sSet(\Delta^m_{\Top},Y)}}
    \end{tikzcd}
  \]
  Taking $g$ to be (a composition of the) $s^i$ 
  gives us that $\widetilde{f_n}$ respects the equivalence relation defining
  $|X_\bullet|$ and hence is a map $\widetilde{f_n}\in \Top(|X_\bullet|, Y)$.
  We can go backwards in a similar fashion.
\end{proof}

\begin{example}
  We define the \textit{standard $n$-simplex} $\Delta^n$ as the simplicial set
  given by letting its $k$-simplices be the set of maps $\Delta([k],[n])$.
\end{example}

\begin{lemma}
  A $k$-simplex $x\in \Delta([k],[n])$ is non-degenerate if and only if
  $x$ is injective.
\end{lemma}
\begin{proof}
  This is immediate once one considers the effect of
  $s_i$ (which is precomposition by $s^i$).
\end{proof}
\begin{lemma}\label{delreal}
  The geometric realisation of $\Delta^n$ is $\Delta^n_{\Top}$. Similarly
  the geometric realisation of $\Delta^n\times \Delta^m$ is
  $\Delta^n_{\Top}\times \Delta^m_{\Top}$.
\end{lemma}
\begin{proof}
  This first statement is a result of \ref{realisation}. The second statement
  also requires \ref{degen}
\end{proof}

\begin{example}
We then define a simplicial set called \textit{the boundary $\partial\Delta^n$
of $\Delta^n$}. We define this by specifying that
$(\partial \Delta^n)_k$ is the set of non-surjective maps
$x\in \mathsf{sSet}([n],[k])$. Indeed $\partial \Delta^n$ is a subsimplex
of $\Delta^n$, and it matches our intuition from
$\Delta_{\Top}^n$ since any element of $(\Delta^n)_k$ which is the image of
some $d^i$ is clearly not surjective. 
\end{example}
\begin{lemma}[The Yoneda Lemma]
  Let $A$ be an object in a category $\mathcal{C}$ and let
  $F: \mathcal{C}\to \mathsf{Set}$ be a functor.
  There is an isomorphism of sets
  $$
  \Natl\left(\mathcal{C}(A,\_ \,), F\right)\cong F(A)
  $$
\end{lemma}
It is not hard to find a proof of this, and even though the lemma
is ubiquitous throughout category theory the proof is very straightforward.
\\
\\
\indent
An immediate consequence of the Yoneda lemma is,
\begin{lemma}
  Given a simplicial set $X_\bullet$ we have that
  $$
  X_n\cong \mathsf{sSet}(\Delta^n, X_\bullet)
  $$
\end{lemma}
\section{Horns and Nerves}
\begin{definition}
  The horn $\Lambda^n_k$ is the simplicial subset of $\partial \Delta^n$ defined
  by
  $$
  (\Lambda^n_k)_i = \left\{ x\in \Delta^n_k\, \big| \, x([k])\cup \{ i \} \supsetneq [n] \right \}
  $$
\end{definition}
Intuitively one thinks of this as the boundary of $\Delta^n$ excluding the $i^{th}$ face.

\begin{definition}
  Let $\mathcal{C}$ be a small category. The \textit{nerve of $\mathcal{C}$} is
  the simplicial set $N\mathcal{C}$ defined by
  $$
  N \mathcal{C}_n =  \left\{ C_0\xrightarrow{f_1}C_1\xrightarrow{f_2}\cdots \xrightarrow{f_n} C_n \right\}
  $$
  The maps $s_i$ and $d_i$ are defined as follows. Let $(f_1,\ldots, f_n)$
  be an ordered $n$-tuple of composable maps in $\mathcal{C}$. We define $s_i$
  to simply be the map $(f_1,\ldots,\id,\ldots,f_n)\in N \mathcal{C}_{n+1}$. The
  map $d_i$ is slightly more complicated.
  \[
      d_i(f_1,\ldots,f_n) =
    \begin{cases}
      (f_2,\ldots,f_n) & \text{if }i=0,\\
      (f_1,\ldots,f_i\circ f_{i+1}, \ldots, f_n) &\text{if }0<i<n,\\
      (f_1,\ldots,f_{n-1}) &\text{if }i=n
    \end{cases}
  \]
\end{definition}

An important example of such a construction is the nerve of the category
$\mathsf{G}$ associated to a group $G$. In this case $N\mathsf{G}_n=G^n$
and thus the geometric realisation of $N \mathcal{C}$ is
$$
\coprod G^n\times \Delta^n_{\Top} \big/ \sim 
$$
This is isomorphic to the Eilenberg Maclane space $K(G,1)$.
\\
\\
\indent
Let $X$ be a simplicial set. We define $M: \msset \to \mathsf{Ch}_{\geq 0}$ by
specifying $M(X)_n$ to the free abelian group generated by the non-degenerate
$n-simplices$ of $X$. Our boundary maps are then the maps $d(x)=\sum_i (-1)^i
d_i(x)$. Another way of stating this is that, by \ref{realisation}, $M(X)$ is the singular chain complex
of $|X|$.

\begin{theorem}\label{main}
  Let $X$ and $Y$ be simplicial sets. Then
  $M(X\times Y)\cong M(X)\otimes M(Y)$
\end{theorem}
This follows from the following theorem,
\begin{theorem}[The Eilenberg-Zilber theorem]
  Given topological spaces $X$, $Y$, and their product $X\times Y$,
  Then $$C_\bullet (X)\otimes C_\bullet (Y) \cong C_\bullet(X\times Y)$$
  in the category of chain complexes.
\end{theorem}
\begin{proof}
  See \cite{EZ}.
\end{proof}
\begin{proof}[Proof of \ref{main}]
  $\left(M(X)\otimes M(Y)\right)_n = \oplus_{i+j=n} M(X)_i \otimes M(Y)_j$.
  By \ref{compos} we have that $(x,y)$ is non-degenerate when either $x$ or
  $y$ is non-degenerate, or when they are incompatibly degenerate. By incompatibly
  degenerate I mean that they decompose as
  $x=s_{i_l}\circ\cdots \circ s_{i_1} x^\prime$ and
  $y=s_{j_k}\circ \cdots \circ s_{j_1} y^\prime$ such that $s_{i_r}\neq s_{j_s}$
  for all $r,s$. Let us refer to $l$ and $k$ from the decomposition above
  as the order of degeneracy for
  arbitrary degenerate elements $x$ and $y$ respectively. A non-degenerate element
  has order of degeneracy 0. To see that the inequality
  holds note that if $(x,y)$ is non-degenerate in $(X\times Y)_n$ then we must have
  that the total order of degeneracy $l+k$ is less than or equal to $n$.
  This follows since if $l+k >n$ then $x$ and $y$ could not possibly be
  incompatibly degenerate by the pigeonhole principle. We may restate this
  succinctly as $(x,y)$ non-degenerate in $X_n\times Y_n$ is
  a pair $(s_{i_l}\circ \cdots s_{i_1} x^\prime, s_{j_k}\circ
  \cdots \circ s_{j_1}y^\prime)$ in $X_n\times Y_n$ where $j_r \neq i_s$ and
  where $x^\prime$ is
  non-degenerate in $X_{n-l}$ and $y^\prime$ is non-degenerate in $Y_{n-k}$,
  where $l+k\leq n$.
  Consider the pair $(x^\prime,y^\prime)$ in $X_{n-l}\times Y_{n-k}$.
  We therefore have that $|X\times Y| = |X| \times |Y|$. This allows us to use
  the Eilenberg-Zilber theorem, which gives us the result (by the discussion
  preceeding \ref{main}).
\end{proof}

\section{\sset  \, as a Model Category}
We can put a model category structure on the category of simplicial sets.
We first define two sets $I= \bigcup_{n\geq 0}\msset (\partial \Delta^n, \Delta^n)$
and $J= \left\{ \msset (\Lambda^n_k, \Delta^n)\, \big| \, n>0
  \text{ and }0\leq k\leq n \right\}$.

\begin{definition}
  Given a map $f\in \msset (X,Y)$, we say $f$ is
  \begin{itemize}
  \item a fibration if $J \boxslash f$, which is to say that if
    we have any map $p\in J$, and any $\Lambda^n_k \to X$
    and $\Delta^n \to Y$, making the following diagram commute,
    \[
    \begin{tikzcd}
      \Lambda^n_k \arrow[r] \arrow[d, "p"] & X \arrow[d, "f"] \\
      \Delta^n \arrow[r] &Y
    \end{tikzcd}
  \]
  then we must have a lift $h: \Delta^n \to X$ which commutes with the diagram,
\item a weak equivalence if $|f|: |X|\to |Y|$ is a weak equivalence in
  \topp,
\item a cofibration if it is injective.
\end{itemize}
\end{definition}
\begin{lemma}
  This defines a model structure on $\msset$.
\end{lemma}
\begin{proof}
  (I think we're showing this next lecture).
\end{proof}
I'm not entirey sure why we introduced the set $I$. I thought it would be used
to define cofibrations but apparently not. Perhaps this will become
apparent next week.


%%%%%%%%%%%%%%%%%%%%%%%%%%%%%%%%%%
%%%%%%%%%%%%%%%%%%%%%%%%%%%%%%%%%%
%%%%%%%%%%%%%%%%%%%%%%%%%%%%%%%%%%
%%%%%%%%%%%%%%%%%%%%%%%%%%%%%%%%%%
% Bibliography %%%%%%%%%%%%%%%%%%%
%%%%%%%%%%%%%%%%%%%%%%%%%%%%%%%%%%
\begin{thebibliography}{10}
\bibitem[May]{May} Jon Peter May, {\it Simplicial Objects in Algebraic Topology}, University of Chicago Press, 1982
\bibitem[EZ]{EZ} Eilenberg, S., \& Zilber, J. (1953). {\it On Products of Complexes}. American Journal of Mathematics, 75(1), 200-204. doi:10.2307/2372629
\bibitem[BK]{BK} Bousfield, A.K., \& Kan, D.M. (1972). {\it Homotopy Limits, Completions and Localizations}. Springer-Verlag Berlin Heidelberg.
\end{thebibliography} 


% \begin{thebibliography}{99}

% \bibitem{adamek} Ad\'{a}mek and Rosicky. {\it Locally Presentable
% and Acessible Categories}. Cambridge University Press, Cambridge,
% 1994.

% \end{thebibliography}

\end{document}
