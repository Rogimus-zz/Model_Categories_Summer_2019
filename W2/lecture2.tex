\documentclass{amsart}

\usepackage{todonotes}
\usepackage{bm}

\def\Z{\mathbb{Z}}
\def\C{\mathbb{C}}
\def\F{\mathbb{F}}

\newcommand{\boxtimes}{\odot}
\newcommand{\stacks}{\Delta}
\newcommand{\toposX}{\mathfrak X}
\newcommand{\etale}{\'{e}tale\,}
\newcommand{\Cech}{\v{C}ech\,}
\newcommand{\bigdot}{\bullet}
\newcommand{\coproduct}{\coprod}
\newcommand{\catname}[1]{{\sffamily\upshape{{#1}}}}
\newcommand{\sset}{\catname{sSet}}
\newcommand{\topp}{\catname{Top}}
\DeclareMathOperator{\Sing}{Sing}
\DeclareMathOperator{\Sub}{\mathcal P}
\DeclareMathOperator{\coeq}{\mathfrak CE}
\DeclareMathOperator{\group}{\mathfrak GPD}
\DeclareMathOperator{\colim}{colim}
\DeclareMathOperator{\Top}{\mathsf{Top}}
%\DeclareMathOperator{\topp}{Top}
%\DeclareMathOperator{\sSet}{sSet}
\DeclareMathOperator{\PTop}{Fun^{\text{pres}}}
\DeclareMathOperator{\holim}{holim}
\DeclareMathOperator{\cosk}{cosk}
\DeclareMathOperator{\hocolim}{hocolim}
\DeclareMathOperator{\bd}{\partial}
\DeclareMathOperator{\U}{\mathcal{U}}
\DeclareMathOperator{\calU}{\mathcal{U}}
\DeclareMathOperator{\calW}{\mathcal{W}}
\DeclareMathOperator{\calE}{\mathcal{E}}
\DeclareMathOperator{\calB}{\mathcal{B}}
\DeclareMathOperator{\calK}{\mathcal{K}}
\DeclareMathOperator{\Sp}{Sp}
\DeclareMathOperator{\Simp}{\bold{CX}}
\DeclareMathOperator{\calF}{\mathcal{F}}
\DeclareMathOperator{\calG}{\mathcal{G}}
\DeclareMathOperator{\Hom}{Hom} \DeclareMathOperator{\bP}{\beta P}
\DeclareMathOperator{\HH}{H} \DeclareMathOperator{\BU}{BU}
\DeclareMathOperator{\id}{id} \DeclareMathOperator{\Fun}{Fun}
\DeclareMathOperator{\calC}{\mathcal{C}}
\DeclareMathOperator{\calI}{\mathcal{I}}
\DeclareMathOperator{\calJ}{\mathcal{J}}
\DeclareMathOperator{\SSet}{\mathcal{S}}
\DeclareMathOperator{\calS}{\mathfrak{S}}
\DeclareMathOperator{\calX}{\mathcal{X}}
\DeclareMathOperator{\red}{\text{hyp}}
\DeclareMathOperator{\calY}{\mathcal{Y}}
\DeclareMathOperator{\op}{op}
\DeclareMathOperator{\calD}{\mathcal{D}}
\DeclareMathOperator{\Ind}{Ind} \DeclareMathOperator{\Acc}{Acc}
\DeclareMathOperator{\Pre}{\PTop}
\DeclareMathOperator{\calP}{\mathcal{P}} \topmargin=0in
\oddsidemargin=0in \evensidemargin=0in \textwidth=6.5in
\textheight=8.5in

\newtheorem{theorem}{Theorem}[subsection]
\newtheorem{lemma}[theorem]{Lemma}
\newtheorem{proposition}[theorem]{Proposition}
\newtheorem{corollary}[theorem]{Corollary}
\newtheorem{fact}[theorem]{Fact}
% \newtheorem{bigtheorem}{Theorem}
\newtheorem{properties}[theorem]{Properties}

\usepackage{tikz-cd}

\theoremstyle{definition}
\newtheorem{definition}[theorem]{Definition}
\newtheorem{example}[theorem]{Example}
\newtheorem{counterexample}[theorem]{Counterexample}
\newtheorem{conjecture}[theorem]{Conjecture}
\newtheorem{postulate}[theorem]{Postulate}
\newtheorem{remark}[theorem]{Remark}

%\newtheorem{notation}[theorem]{Notation}
%\newtheorem{question}[theorem]{Question}
%\newtheorem{variant}[theorem]{Variant}


\begin{document}

\title{Lecture 2 -- Simplicial Sets}
\author{notes anotated by Roger Murray}
\date{\today}

\maketitle
\section{Notation}
Categories will typically be sans serif (\textit{i.e.} \topp, \sset, etc.).
If $\calC$ is an arbitrary category then we denote the the set of $\calC$-morphisms from
objects $A,B\in \calC$ by $\mathcal{C}(A,B)$.

\section{Garbage}

Recall the definition of the \textit{standard topological $n$-simplex} as the set
$$
\Delta^n_{\Top} := \left\{ (t_0,\cdots, t_n)\in \mathbb{R}^{n+1} \, \Big| \,
  \sum t_i = 1 \text{ and } t_i\geq 0 \text{ for all }i \right\}
$$
Alternatively we may think of $\Delta^n_{\Top}$ as the convex hull of
vertices $v_i=(0,\ldots,1,\ldots,0)$.
We then have maps codegeneracy maps $s^i:\Delta^{n+1}_{\Top} \to \Delta^n_{\Top}$,
and coface maps $d^i: \Delta^{n-1}_{\Top} \to \Delta^n_{\Top}$ defined by
\begin{align*}
  & s^i(t_0,\ldots,t_{n+1}) = (t_0,\ldots,t_i+t_{i+1},\ldots,t_{n+1}) \\
  & d^i(t_0,\ldots,t_{n-1}) = (t_0,\ldots,t_i,0,t_{i+1},\ldots,t_n)
\end{align*}
Clearly $d^i$ is just the map embedding $\Delta^{n-1}_{\Top}$ as the $i^{th}$ face of $\Delta^{n}_{\Top}$, and $s^i$ is a retration of $\Delta^{n+1}_{\Top}$ minus the $i^{th}$ vertice $v_i$ onto
the face opposite $v_i$.
\\
\\
\indent
Given any topological space $X\in\Top$ we define the \textit{singular $n$-simplices of $X$} to be the maps
$$
\Sing(X)_n := \Top( \Delta^n_{\Top},\, X)
$$
This turns out to be an important example of something called a simplicial object.
Before we can define what a simplicial object is, we must first define the
\textit{simplex category} $\Delta$. The objects of $\Delta$ are the ordered sets
$[n]=\{0,1,\ldots,n\}$, and the morphisms $f:[m]\to[n]$ are the
weakly-order-preserving (\textit{i.e.} non-decreasing) functions. Similarly to
above we have maps $s^i:[n+1]\to [n]$ and $d^i:[n-1]\to [n]$ given by
repeating the $i$, and skipping $i$ respectively. The following lemma says
that these maps are the only maps we care about.
\begin{lemma}\label{compos}
  Any map $f$ in $\Delta$ is the composition of a series
  of $d^i$ and $s^j$.
\end{lemma}
\begin{proof}[Proof sketch]
  If you have a map $f:[m]\to[n]$ then you have the inequality $f(0)\leq f(1) \leq \cdots \leq f(m)$.
  We get unique elements of the form $g_0<\cdots < g_k$ for $k\leq \min \{m,n\}$. By composing
  $d^0$ with itself $g_0$ times we get a map that sends $[m]$ to $[g_0,g_0+1,\ldots,g_0+m]$.
  We repeat $g_0$ with $s^0$ as many times as it occurs in the sequence of $f(i)$s. We then simply
  repeat this process inductively on the $g_i$ for $0 < i\leq k$.
\end{proof}


\begin{definition} Let $\mathcal{C}$ be a category. A \textit{simplicial object
  in $\mathcal{C}$} is a functor
  $X: \Delta^{\op} \to \mathcal{C}$.
  %\quad \quad\quad \quad \quad \quad \quad \,
  Similarly a \textit{cosimplicial object in $\mathcal{C}$} is a functor
  $Y:\Delta \to \mathcal{C}$.
\end{definition}
We write $X_n$ and $Y^n$ for $X([n])$ and $Y([n])$ respectively, and hence we
will often use the notation $X_\bullet$ and $Y^\bullet$ for $X$ and
$Y$ respectively.
Similarly we write $d_i,s_i$ and $d^i,s^i$ for the obvious maps in $\mathcal{C}$.


\begin{example}
  The standard simplices $\Delta^\bullet_{\Top}$ is a cosimplicial space.
\end{example}
\begin{example}
  Given any space $X\in \Top$ we therefore have that
  $\Sing(X)_\bullet$ is a simplicial set acting on objects by
  $[n]\mapsto \Top(\Delta^n_{\Top}, \, X)$, and on morphisms by
  $(f:[m]\to [n])\mapsto f_*$ where $f_*$ is precomposition
  with $f$.
\end{example}

\begin{properties}\label{simplprops}
  The degeneracy, and face maps satisfy the
  \textit{simplicial properties}:
  
  \begin{enumerate}
  \item $d_i\circ d_j = d_{j-1} \circ d_i$ for $i<j$
  \item $s_i\circ s_j = s_{j+1}\circ s_i$ for $i\leq j$
  \item Lastly, 
    \begin{flalign*}
      d_i\circ s_j = 
     \begin{cases}
       s_{j-1}\circ d_i &\text{ if } i<j\\
       \id &\text{ if } i=j,j+1 \\
       s_j\circ d_{i-1} &\text{ if } i>j+1
     \end{cases}
    \end{flalign*}
  \end{enumerate}
\end{properties}
A dual statement holds for cosimplicial objects. These (dual) properties
are obvious in the case of $\Delta$.
We next construct a functor which is adjoint to $\Sing(\,\text{--}\,)_\bullet$.
\\
\\
\indent
Let $X$ be a simplicial object in a category $\mathcal{C}$.
\begin{definition}
  An $n$-simplex $x\in X_n$ is \textit{degenerate} if there exists
  some $y\in X_{n-1}$ such that $x=s_i(y)$ for some $0\leq i\leq n-1$.
\end{definition}
\begin{definition}
  The \textit{geometric realisation of $X$} is the space given by
  $$
  |X_\bullet| = \coprod_{n\geq 0} X_n\times \Delta^n  \,\Big/\sim
  $$
  where the equivalence relation is given by identifying
  $(f^*x,u)\sim (y,f_*u)$ for any $f\in \Delta([m],[n])$, and
  $x\in X_n$, and $u\in \Delta^m$.
\end{definition}

It turns out that it is enough to consider when $f$ is a degeneracy, or face map.
This is an obvious result from \ref{compos} and the following lemma,

\begin{lemma}
  Every object $x\in X_n$ may be written uniquely as
  $s_{i_1}\circ s_{i_2} \circ \cdots \circ s_{i_l} (y)$ such that
  $i_1>\cdots > i_n$ and $y\in X_\bullet$ is non-degenerate.
\end{lemma}
\begin{proof}
  By (2) of \ref{simplprops} it is easy to see that we can find an
  increasing sequence $i_j$. Furthermore this sequence must terminate since
  $x\in X_n$ and each degeneracy map increases degree by $1$. Thus we obtain
  a $y$ such that $x=s_{i_1}\circ \cdots \circ s_{i_n} (y)$. All that is
  left to check is that these are unique. This is immediate by starting
  at $x$ and working backwards. 
\end{proof}

\begin{corollary}
  Given a simplicial object $X$ in $\mathcal{C}$ then its geometric
  realisation $|X|$ is a CW-complex with an $n$-cell for every non-degenerate
  $n$-simplex $x\in X_n$.
\end{corollary}
For a proof of this see page 56 of \cite{May}.

\begin{proposition}
  The functors
  \begin{tikzcd}
    {|\text{ -- }|: \mathsf{sSet}} \arrow[r, shift right] &
    {\arrow[l, shift right] \Top : \Sing}
  \end{tikzcd}
are adjoint.
\end{proposition}
\begin{proof}
  We wish to show that, given any $X_\bullet\in \mathsf{sSet}$ and
  any $Y\in \Top$, the sets
  $\mathsf{sSet}(X_\bullet, \Sing(Y)_\bullet)$ and
  $\Top(|X_\bullet|, Y)$ are in bijection.
  This proof relies on the fact that any map $f_n\in \mathsf{sSet}(X_n,
  \Sing(Y)_n)$ corresponds to a map
  $\widetilde{f_n}: X_n\times \Delta^n_{\Top}\to Y$ where $\widetilde{f_n}(x,u) = f_n(x)(u)$.
  If we now consider any map $g\in \Delta([m],[n])$ then we get the following commutative
  diagram by definition
  \[
    \begin{tikzcd}
      X_n \arrow[r,"f_n"] \arrow[d,"X(g)"] & {\mathsf{sSet(\Delta^n_{\Top},Y)}}
      \arrow[d, "\left(\Delta^\bullet_{\Top} (g)\right)^*"]\\
      X_m \arrow[r, "f_m"] &  {\mathsf{sSet(\Delta^m_{\Top},Y)}}
    \end{tikzcd}
  \]
  By the existence of $\widetilde{f_n}$ in relation to $f_n$ this gives us the
  following commutative diagram
  \[
    \begin{tikzcd}
      X_n\times \Delta^m_{Top} \arrow[r,"\id \times \Delta_{\Top}^\bullet(g)"]
      \arrow[d,"X(g)\times \id"]
      & X_n\times \Delta^n_{\Top}
      \arrow[d, "\widetilde{f_n}"]\\
      X_m \times \Delta^m_{\Top} \arrow[r, "\widetilde{f_m}"]
      &  {\mathsf{sSet(\Delta^m_{\Top},Y)}}
    \end{tikzcd}
  \]
  Taking $g$ to be (a composition of the) $s^i$ 
  gives us that $\widetilde{f_n}$ respects the equivalence relation defining
  $|X_\bullet|$ and hence is a map $\widetilde{f_n}\in \Top(|X_\bullet|, Y)$.
  We can go backwards in a similar fashion.
\end{proof}



%%%%%%%%%%%%%%%%%%%%%%%%%%%%%%%%%%
%%%%%%%%%%%%%%%%%%%%%%%%%%%%%%%%%%
%%%%%%%%%%%%%%%%%%%%%%%%%%%%%%%%%%
%%%%%%%%%%%%%%%%%%%%%%%%%%%%%%%%%%
% Bibliography %%%%%%%%%%%%%%%%%%%
%%%%%%%%%%%%%%%%%%%%%%%%%%%%%%%%%%
\begin{thebibliography}{10}
  \bibitem{May} Jon Peter May, {\it Simplicial Objects in Algebraic Topology}, University of Chicago Press, 1982
\end{thebibliography}


% \begin{thebibliography}{99}

% \bibitem{adamek} Ad\'{a}mek and Rosicky. {\it Locally Presentable
% and Acessible Categories}. Cambridge University Press, Cambridge,
% 1994.

% \end{thebibliography}

\end{document}
