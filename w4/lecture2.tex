\documentclass{amsart}
\usepackage{todonotes}
\usepackage{amssymb}
\usepackage{mathabx}
\usepackage{mathtools}
\usepackage{pigpen}

\def\Z{\mathbb{Z}}
\def\C{\mathbb{C}}
\def\F{\mathbb{F}}

%\newcommand{\boxtimes}{\odot}
\newcommand{\stacks}{\Delta}
\newcommand{\toposX}{\mathfrak X}
\newcommand{\etale}{\'{e}tale\,}
\newcommand{\Cech}{\v{C}ech\,}
\newcommand{\bigdot}{\bullet}
\newcommand{\coproduct}{\coprod}
\newcommand{\catname}[1]{{\sffamily\upshape{{#1}}}}
\newcommand{\sset}{\catname{sSet}}
\newcommand{\topp}{\catname{Top}}
\DeclareMathOperator{\cell}{cell}
\DeclareMathOperator{\Sing}{Sing}
\DeclareMathOperator{\Sub}{\mathcal P}
\DeclareMathOperator{\coeq}{\mathfrak CE}
\DeclareMathOperator{\group}{\mathfrak GPD}
\DeclareMathOperator{\colim}{colim}
\DeclareMathOperator{\Top}{\mathsf{Top}}
\DeclareMathOperator{\Natl}{Natl}
%\DeclareMathOperator{\topp}{Top}
\DeclareMathOperator{\msset}{\mathsf{sSet}}
\DeclareMathOperator{\PTop}{Fun^{\text{pres}}}
\DeclareMathOperator{\holim}{holim}
\DeclareMathOperator{\cosk}{cosk}
\DeclareMathOperator{\hocolim}{hocolim}
\DeclareMathOperator{\bd}{\partial}
\DeclareMathOperator{\U}{\mathcal{U}}
\DeclareMathOperator{\calU}{\mathcal{U}}
\DeclareMathOperator{\calW}{\mathcal{W}}
\DeclareMathOperator{\calE}{\mathcal{E}}
\DeclareMathOperator{\calB}{\mathcal{B}}
\DeclareMathOperator{\calK}{\mathcal{K}}
\DeclareMathOperator{\Sp}{Sp}
\DeclareMathOperator{\Simp}{\bold{CX}}
\DeclareMathOperator{\calF}{\mathcal{F}}
\DeclareMathOperator{\calG}{\mathcal{G}}
\DeclareMathOperator{\Hom}{Hom} \DeclareMathOperator{\bP}{\beta P}
\DeclareMathOperator{\HH}{H} \DeclareMathOperator{\BU}{BU}
\DeclareMathOperator{\id}{id} \DeclareMathOperator{\Fun}{Fun}
\DeclareMathOperator{\calC}{\mathcal{C}}
\DeclareMathOperator{\calI}{\mathcal{I}}
\DeclareMathOperator{\calJ}{\mathcal{J}}
\DeclareMathOperator{\SSet}{\mathcal{S}}
\DeclareMathOperator{\calS}{\mathfrak{S}}
\DeclareMathOperator{\calX}{\mathcal{X}}
\DeclareMathOperator{\red}{\text{hyp}}
\DeclareMathOperator{\calY}{\mathcal{Y}}
\DeclareMathOperator{\op}{op}
\DeclareMathOperator{\calD}{\mathcal{D}}
\DeclareMathOperator{\Ind}{Ind} \DeclareMathOperator{\Acc}{Acc}
\DeclareMathOperator{\Pre}{\PTop}
\DeclareMathOperator{\calP}{\mathcal{P}} \topmargin=0in
\oddsidemargin=0in \evensidemargin=0in \textwidth=6.5in
\textheight=8.5in


\tikzset{
    labln/.style={anchor=north, rotate=90, inner sep=.5mm}
  }
  \tikzset{
    labls/.style={anchor=south, rotate=90, inner sep=.5mm}
  }

\newtheorem{theorem}{Theorem}[subsection]
\newtheorem{lemma}[theorem]{Lemma}
\newtheorem{proposition}[theorem]{Proposition}
\newtheorem{corollary}[theorem]{Corollary}
\newtheorem{fact}[theorem]{Fact}
% \newtheorem{bigtheorem}{Theorem}
\newtheorem{properties}[theorem]{Properties}

\usepackage{tikz-cd}

\theoremstyle{definition}
\newtheorem{definition}[theorem]{Definition}
\newtheorem{example}[theorem]{Example}
\newtheorem{counterexample}[theorem]{Counterexample}
\newtheorem{conjecture}[theorem]{Conjecture}
\newtheorem{postulate}[theorem]{Postulate}
\newtheorem{remark}[theorem]{Remark}

\newtheorem{notation}[theorem]{Notation}
%\newtheorem{question}[theorem]{Question}
%\newtheorem{variant}[theorem]{Variant}


\begin{document}

\title{Lecture 2 -- Simplicial Sets}
\author{notes anotated by Roger Murray}
\date{\today}

\maketitle
\section{Notation}
Categories will typically be sans serif (\textit{i.e.} \topp, \sset, etc.) or caligraphric
(\textit{i.e.} $\mathcal{C}$, $\mathcal{S}$, etc.).
If $\calC$ is an arbitrary category then we denote the the set of $\calC$-morphisms from
objects $A,B\in \mathcal{C}$ by $\mathcal{C}(A,B)$.

\section{comments for readers}
This is a set of notes based on lectures given by Vigleik Angeltveit at
the Australian National University in the (AUS) summer of 2019.
For quite a few people who are taking this course this is their first
early-graduate level course in algebraic topology, and hence I have
tried to be very explicit (read: overly explicit) in much of the exposition.
Furthermore, due to the pace of the course, many of the proofs are simplified
or omitted. Where this has occured I have either attempted to generate my own
proof, or I have replicated arguments from places such as \cite{Hov},
[LURIE...]
or nLab. Finally, since most people attending the class have complete
board notes from the lectures, in order for these notes to be somewhat useful
(for myself as writing practise, and for the students taking this course)
I have inserted my own exposition and explanation throughout, and
I may slightly deviate from Vigleik's own notes in certain places
based on my own personal preferences.

These notes are constantly being updated as the course won't be finished
until at least the end of January, and I encourage you to check in regularly. 
\section{model categories and some classical examples}
\subsection{Model categories}
\begin{definition}
  A \textit{model category} is a category $\mathcal{C}$ together with 3 distinct 
  classes of maps - fibrations, cofirbrations and weak equivalences - 
  satisfying  5 axioms. If we denote any fibration by $A\twoheadrightarrow B$,
  any cofibration by $A\rightarrowtail B$, and any weak equivalence by
  $A\xrightarrow{\sim}B$, then these axioms may be stated as follows:

  \begin{enumerate}
  \item[(M1)] $\mathcal{C}$ is bicomplete
  \item[(M2)] Weak equivalences satisfy the 2/3 axiom, \textit{i.e.} given 
    any diagram in $\mathcal{C}$ of the form
    \[
    \begin{tikzcd}
      A\arrow[rr] \arrow[dr] & & B\\
      &C\arrow[ur]
    \end{tikzcd}
    \] such that any two of the maps are weak equivalences, then the third must be
    one also.
  \item[(M3)] Any map $f\in \mathcal{C}(A,B)$ can be factored for some $C\in \mathcal{C}$
    as either
    \[
      \begin{tikzcd}
        A\arrow[r, tail, "\sim", "g"'] & C \arrow[r, twoheadrightarrow, "h"'] & B & \text{ or } &
        A \arrow[r, tail, "g"] & C\arrow[r, twoheadrightarrow, "\sim", "h"'] & B  
      \end{tikzcd}
    \]
  \item[(M4)]
    Given either of the following commutative squares,
    \[
      \begin{tikzcd}
        A\arrow[r]\arrow[d, tail, "\sim" labln, "g"'] &   X\arrow[d,twoheadrightarrow, "f"]
        & & A\arrow[r]\arrow[d, tail, "g"'] &   X\arrow[d,twoheadrightarrow, "\sim" labls, "f"]
        \\ B\arrow[r] &Y & \text{or} & B\arrow[r] &Y
      \end{tikzcd}
    \]
    then there is a (not necessarily unique) lift $B\to X$ which commutes with the given diagram.

  \item[(M5)]
    If $f$ is a retract of $g$, and if $g$ is a fibration/cofibration/weak-equivalence
    then $f$ is also. 
  
  \end{enumerate}
\end{definition}
\noindent
Recall that we say $f:A\to B$ is a retract of $g: C\to D$ if their exist maps $i,r, j,s$
making the following diagram commute,
\[
  \begin{tikzcd}
    A \arrow[r, "i"]\arrow[d,"f"]\arrow[rr,bend left=30, "\id"] & C\arrow[r,"r"]\arrow[d,"g"] & A \arrow[d, "f"] \\
    B \arrow[r, "j"] \arrow[rr,bend right=30, "\id"']& B \arrow[r, "s"]& B
  \end{tikzcd}
\]
\begin{example}
  The proto-typical example is \topp \, with fibrations the Serre fibrations,
  and cofibrations and weak-equivalences decided accordingly (see chapter 4 of
  \cite{Hat}).
\end{example}

\subsection{Simplicies and simlpicial sets}

Recall the definition of the \textit{standard topological $n$-simplex} as the set
$$
\Delta^n_{\Top} := \left\{ (t_0,\cdots, t_n)\in \mathbb{R}^{n+1} \, \Big| \,
  \sum t_i = 1 \text{ and } t_i\geq 0 \text{ for all }i \right\}
$$
Alternatively we may think of $\Delta^n_{\Top}$ as the convex hull of
vertices $v_i=(0,\ldots,1,\ldots,0)$.
We then have \textit{codegeneracy maps} $s^i:\Delta^{n+1}_{\Top} \to \Delta^n_{\Top}$,
and \textit{coface maps} $d^i: \Delta^{n-1}_{\Top} \to \Delta^n_{\Top}$ defined by
\begin{align*}
  & s^i(t_0,\ldots,t_{n+1}) = (t_0,\ldots,t_i+t_{i+1},\ldots,t_{n+1}) \\
  & d^i(t_0,\ldots,t_{n-1}) = (t_0,\ldots,t_i,0,t_{i+1},\ldots,t_n)
\end{align*}
Clearly $d^i$ is just the map embedding $\Delta^{n-1}_{\Top}$ as the $i^{th}$ face of $\Delta^{n}_{\Top}$, and $s^i$ is a retration of $\Delta^{n+1}_{\Top}$ minus the $i^{th}$ vertice $v_i$ onto
the face opposite $v_i$.
\\
\\
\indent
Given any topological space $X\in\Top$ we define the \textit{singular $n$-simplices of $X$} to be the set of maps
$$
\Sing(X)_n := \Top( \Delta^n_{\Top},\, X)
$$
This turns out to be an important example of something called a simplicial object.
Before we can define what a simplicial object is, we must first define the
\textit{simplex category} $\Delta$. The objects of $\Delta$ are the ordered sets
$[n]=\{0,1,\ldots,n\}$, and the morphisms $f:[m]\to[n]$ are the
weakly-order-preserving (\textit{i.e.} non-decreasing) functions. Similarly to
above we have maps $s^i:[n+1]\to [n]$ and $d^i:[n-1]\to [n]$ given by
repeating the $i$, and skipping $i$ respectively. The following lemma says
that these maps are the only maps we care about.
\begin{lemma}\label{compos}
  Any map $f$ in $\Delta$ is the composition of some $d^i$ and $s^j$.
\end{lemma}
\begin{proof}[Proof sketch]
  If you have a map $f:[m]\to[n]$ then you have the inequality $f(0)\leq f(1) \leq \cdots \leq f(m)$.
  We obtain a unique chain of elements of the form $g_0<\cdots < g_k$
  for $k\leq \min \{m,n\}$. By composing
  $d^0$ with itself $g_0$ times we get a map that sends $[m]$ to $[g_0,g_0+1,\ldots,g_0+m]$.
  We repeat $g_0$ with $s^0$ as many times as it occurs in the sequence of $f(i)$s. We then simply
  repeat this process inductively on the $g_i$ for $0 < i\leq k$.
\end{proof}


\begin{definition} Let $\mathcal{C}$ be a category. A \textit{simplicial object
  in $\mathcal{C}$} is a functor
  $X: \Delta^{\op} \to \mathcal{C}$.
  %\quad \quad\quad \quad \quad \quad \quad \,
  Similarly a \textit{cosimplicial object in $\mathcal{C}$} is a functor
  $Y:\Delta \to \mathcal{C}$.
\end{definition}
We write $X_n$ and $Y^n$ for $X([n])$ and $Y([n])$ respectively, and hence we
will often use the notation $X_\bullet$ and $Y^\bullet$ for $X$ and
$Y$ respectively.
Similarly we write $d_i,s_i$ and $d^i,s^i$ for the obvious maps in $\mathcal{C}$.


\begin{example}
  The standard simplices $\Delta^\bullet_{\Top}$ is a cosimplicial space.
\end{example}
\begin{example}
  Given any space $X\in \Top$ we therefore have that
  $\Sing(X)_\bullet$ is a simplicial set acting on objects by
  $[n]\mapsto \Top(\Delta^n_{\Top}, \, X)$, and on morphisms by
  $(f:[m]\to [n])\mapsto f_*$ where $f_*$ is precomposition
  with $f$.
\end{example}

\begin{properties}\label{simplprops}
  The degeneracy, and face maps belonging to any simplicial object $X$ in
  $\mathcal{C}$ satisfy the following,
  
  \begin{enumerate}
  \item $d_i\circ d_j = d_{j-1} \circ d_i$ for $i<j$
  \item $s_i\circ s_j = s_{j+1}\circ s_i$ for $i\leq j$
  \item Lastly, 
    \begin{flalign*}
      d_i\circ s_j = 
     \begin{cases}
       s_{j-1}\circ d_i &\text{ if } i<j\\
       \id &\text{ if } i=j,j+1 \\
       s_j\circ d_{i-1} &\text{ if } i>j+1
     \end{cases}
    \end{flalign*}
  \end{enumerate}
\end{properties}
We refer to these properties as the \textit{simplicial properties}.
A dual statement holds for cosimplicial objects. These (dual) properties
are obvious in the case of $\Delta$.
We next construct a functor which is adjoint to $\Sing(\,\text{--}\,)_\bullet$.
\\
\\
\indent
Let $X$ be a simplicial object in a category $\mathcal{C}$.
\begin{definition}
  An $n$-simplex $x\in X_n$ is \textit{degenerate} if there exists
  some $y\in X_{n-1}$ such that $x=s_i(y)$ for some $0\leq i\leq n-1$.
\end{definition}
\begin{definition}
  The \textit{geometric realisation of $X$} is the space given by
  $$
  |X_\bullet| = \coprod_{n\geq 0} X_n\times \Delta^n_{\Top}  \,\Big/\sim
  $$
  where the equivalence relation is given by identifying
  $(f^*x,u)\sim (y,f_*u)$ for any $f\in \Delta([m],[n])$, and
  $x\in X_n$, and $u\in \Delta^m_{\Top}$.
\end{definition}

By \ref{compos} it is enough to consider when $f$ is a degeneracy, or face map.
\\
\\
\indent
We have the following useful lemma,
\begin{lemma}\label{degen}
  Every object $x\in X_n$ may be written uniquely as
  $s_{i_1}\circ s_{i_2} \circ \cdots \circ s_{i_l} (y)$ such that
  $i_1>\cdots > i_n$ and $y\in X_\bullet$ is non-degenerate.
\end{lemma}
\begin{proof}
  By (2) of \ref{simplprops} it is easy to see that we can find an
  increasing sequence $i_j$. Furthermore this sequence must terminate since
  $x\in X_n$ and each degeneracy map increases degree by $1$. Thus we obtain
  a $y$ such that $x=s_{i_1}\circ \cdots \circ s_{i_n} (y)$. All that is
  left to check is that these are unique. This is immediate by starting
  at $x$ and working backwards. 
\end{proof}

Given two simplicial sets $X$ and $Y$ we obtain another simplicial set
$X\times Y$ defined by $(X\times Y)_n = X_n\times Y_n$. The above lemma
implies that we may combine two degenerate simplices $x\in X_n$
and $y\in Y_n$ to get a non-degenerate simplex $(x,y)$. This will be important
below.

\begin{lemma}\label{realisation}
  Given a simplicial set $X$ its geometric
  realisation $|X|$ is a CW-complex with an $n$-cell for every non-degenerate
  $n$-simplex $x\in X_n$.
\end{lemma}


For a classical proof of this see page 56 of \cite{May}.



\begin{proposition}
  The functors
  \begin{tikzcd}
    {|\text{ -- }|: \mathsf{sSet}} \arrow[r, shift right] &
    {\arrow[l, shift right] \Top : \Sing}
  \end{tikzcd}
are adjoint.
\end{proposition}
\begin{proof}
  We wish to show that, given any $X_\bullet\in \mathsf{sSet}$ and
  any $Y\in \Top$, the sets
  $\mathsf{sSet}(X_\bullet, \Sing(Y)_\bullet)$ and
  $\Top(|X_\bullet|, Y)$ are in bijection.
  This proof relies on the fact that any map $f_n\in \mathsf{sSet}(X_n,
  \Sing(Y)_n)$ corresponds to a map
  $\widetilde{f_n}: X_n\times \Delta^n_{\Top}\to Y$ where $\widetilde{f_n}(x,u) = f_n(x)(u)$.
  If we now consider any map $g\in \Delta([m],[n])$ then we get the following commutative
  diagram by definition
  \[
    \begin{tikzcd}
      X_n \arrow[r,"f_n"] \arrow[d,"X(g)"] & {\mathsf{sSet(\Delta^n_{\Top},Y)}}
      \arrow[d, "\left(\Delta^\bullet_{\Top} (g)\right)^*"]\\
      X_m \arrow[r, "f_m"] &  {\mathsf{sSet(\Delta^m_{\Top},Y)}}
    \end{tikzcd}
  \]
  By the existence of $\widetilde{f_n}$ in relation to $f_n$ this gives us the
  following commutative diagram
  \[
    \begin{tikzcd}
      X_n\times \Delta^m_{\Top} \arrow[r,"\id \times \Delta_{\Top}^\bullet(g)"]
      \arrow[d,"X(g)\times \id"]
      & X_n\times \Delta^n_{\Top}
      \arrow[d, "\widetilde{f_n}"]\\
      X_m \times \Delta^m_{\Top} \arrow[r, "\widetilde{f_m}"]
      &  {\mathsf{sSet(\Delta^m_{\Top},Y)}}
    \end{tikzcd}
  \]
  Taking $g$ to be (a composition of the) $s^i$ 
  gives us that $\widetilde{f_n}$ respects the equivalence relation defining
  $|X_\bullet|$ and hence is a map $\widetilde{f_n}\in \Top(|X_\bullet|, Y)$.
  We can go backwards in a similar fashion.
\end{proof}

\begin{example}
  We define the \textit{standard $n$-simplex} $\Delta^n$ as the simplicial set
  given by letting its $k$-simplices be the set of maps $\Delta([k],[n])$.
\end{example}

\begin{lemma}
  A $k$-simplex $x\in \Delta([k],[n])$ is non-degenerate if and only if
  $x$ is injective.
\end{lemma}
\begin{proof}
  This is immediate once one considers the effect of
  $s_i$ (which is precomposition by $s^i$).
\end{proof}
\begin{lemma}\label{delreal}
  The geometric realisation of $\Delta^n$ is $\Delta^n_{\Top}$. Similarly
  the geometric realisation of $\Delta^n\times \Delta^m$ is
  $\Delta^n_{\Top}\times \Delta^m_{\Top}$.
\end{lemma}
\begin{proof}
  This first statement is a result of \ref{realisation}. The second statement
  also requires \ref{degen}
\end{proof}

\begin{example}
We then define a simplicial set called \textit{the boundary $\partial\Delta^n$
of $\Delta^n$}. We define this by specifying that
$(\partial \Delta^n)_k$ is the set of non-surjective maps
$x\in \mathsf{sSet}([n],[k])$. Indeed $\partial \Delta^n$ is a subsimplex
of $\Delta^n$, and it matches our intuition from
$\Delta_{\Top}^n$ since any element of $(\Delta^n)_k$ which is the image of
some $d^i$ is clearly not surjective. 
\end{example}
\begin{lemma}[The Yoneda Lemma]
  Let $A$ be an object in a category $\mathcal{C}$ and let
  $F: \mathcal{C}\to \mathsf{Set}$ be a functor.
  There is an isomorphism of sets
  $$
  \Natl\left(\mathcal{C}(A,\_ \,), F\right)\cong F(A)
  $$
\end{lemma}
It is not hard to find a proof of this, and even though the lemma
is ubiquitous throughout category theory the proof is very straightforward.
\\
\\
\indent
An immediate consequence of the Yoneda lemma is,
\begin{lemma}
  Given a simplicial set $X_\bullet$ we have that
  $$
  X_n\cong \mathsf{sSet}(\Delta^n, X_\bullet)
  $$
\end{lemma}
\section{horns and nerves}
\begin{definition}
  The horn $\Lambda^n_k$ is the simplicial subset of $\partial \Delta^n$ defined
  by
  $$
  (\Lambda^n_k)_i = \left\{ x\in \Delta^n_k\, \big| \, x([k])\cup \{ i \} \supsetneq [n] \right \}
  $$
\end{definition}
Intuitively one thinks of this as the boundary of $\Delta^n$ excluding the $i^{th}$ face.

\begin{definition}
  Let $\mathcal{C}$ be a small category. The \textit{nerve of $\mathcal{C}$} is
  the simplicial set $N\mathcal{C}$ defined by
  $$
  N \mathcal{C}_n =  \left\{ C_0\xrightarrow{f_1}C_1\xrightarrow{f_2}\cdots \xrightarrow{f_n} C_n \right\}
  $$
  The maps $s_i$ and $d_i$ are defined as follows. Let $(f_1,\ldots, f_n)$
  be an ordered $n$-tuple of composable maps in $\mathcal{C}$. We define $s_i$
  to simply be the map $(f_1,\ldots,\id,\ldots,f_n)\in N \mathcal{C}_{n+1}$. The
  map $d_i$ is slightly more complicated.
  \[
      d_i(f_1,\ldots,f_n) =
    \begin{cases}
      (f_2,\ldots,f_n) & \text{if }i=0,\\
      (f_1,\ldots,f_i\circ f_{i+1}, \ldots, f_n) &\text{if }0<i<n,\\
      (f_1,\ldots,f_{n-1}) &\text{if }i=n
    \end{cases}
  \]
\end{definition}

An important example of such a construction is the nerve of the category
$\mathsf{G}$ associated to a group $G$. In this case $N\mathsf{G}_n=G^n$
and thus the geometric realisation of $N \mathcal{C}$ is
$$
\coprod G^n\times \Delta^n_{\Top} \big/ \sim 
$$
This is isomorphic to the Eilenberg Maclane space $K(G,1)$.
\\
\\
\indent
Let $X$ be a simplicial set. We define $M: \msset \to \mathsf{Ch}_{\geq 0}$ by
specifying $M(X)_n$ to the free abelian group generated by the non-degenerate
$n-simplices$ of $X$. Our boundary maps are then the maps $d(x)=\sum_i (-1)^i
d_i(x)$. Another way of stating this is that, by \ref{realisation}, $M(X)$ is the singular chain complex
of $|X|$.

\begin{theorem}\label{main}
  Let $X$ and $Y$ be simplicial sets. Then
  $M(X\times Y)\cong M(X)\otimes M(Y)$
\end{theorem}
This follows from the following theorem,
\begin{theorem}[The Eilenberg-Zilber theorem]
  Given topological spaces $X$, $Y$, and their product $X\times Y$,
  Then $$C_\bullet (X)\otimes C_\bullet (Y) \cong C_\bullet(X\times Y)$$
  in the category of chain complexes.
\end{theorem}
\begin{proof}
  See \cite{EZ}.
\end{proof}
\begin{proof}[Proof of \ref{main}]
  $\left(M(X)\otimes M(Y)\right)_n = \oplus_{i+j=n} M(X)_i \otimes M(Y)_j$.
  By \ref{compos} we have that $(x,y)$ is non-degenerate when either $x$ or
  $y$ is non-degenerate, or when they are incompatibly degenerate. By incompatibly
  degenerate I mean that they decompose as
  $x=s_{i_l}\circ\cdots \circ s_{i_1} x^\prime$ and
  $y=s_{j_k}\circ \cdots \circ s_{j_1} y^\prime$ such that $s_{i_r}\neq s_{j_s}$
  for all $r,s$. Let us refer to $l$ and $k$ from the decomposition above
  as the order of degeneracy for
  arbitrary degenerate elements $x$ and $y$ respectively. A non-degenerate element
  has order of degeneracy 0. To see that the inequality
  holds note that if $(x,y)$ is non-degenerate in $(X\times Y)_n$ then we must have
  that the total order of degeneracy $l+k$ is less than or equal to $n$.
  This follows since if $l+k >n$ then $x$ and $y$ could not possibly be
  incompatibly degenerate by the pigeonhole principle. We may restate this
  succinctly as $(x,y)$ non-degenerate in $X_n\times Y_n$ is
  a pair $(s_{i_l}\circ \cdots s_{i_1} x^\prime, s_{j_k}\circ
  \cdots \circ s_{j_1}y^\prime)$ in $X_n\times Y_n$ where $j_r \neq i_s$ and
  where $x^\prime$ is
  non-degenerate in $X_{n-l}$ and $y^\prime$ is non-degenerate in $Y_{n-k}$,
  where $l+k\leq n$.
  Consider the pair $(x^\prime,y^\prime)$ in $X_{n-l}\times Y_{n-k}$.
  We therefore have that $|X\times Y| = |X| \times |Y|$. This allows us to use
  the Eilenberg-Zilber theorem, which gives us the result (by the discussion
  preceeding \ref{main}).
\end{proof}

\section{\sset  \, as a model category}
We can put a model category structure on the category of simplicial sets.
We first define two sets $I= \bigcup_{n\geq 0}\msset (\partial \Delta^n, \Delta^n)$
and $J= \left\{ \msset (\Lambda^n_k, \Delta^n)\, \big| \, n>0
  \text{ and }0\leq k\leq n \right\}$.

\begin{definition}
  Given a map $f\in \msset (X,Y)$, we say $f$ is
  \begin{itemize}
  \item a fibration if $J \boxslash f$, which is to say that if
    we have any map $p\in J$, and any $\Lambda^n_k \to X$
    and $\Delta^n \to Y$, making the following diagram commute,
    \[
    \begin{tikzcd}
      \Lambda^n_k \arrow[r] \arrow[d, "p"] & X \arrow[d, "f"] \\
      \Delta^n \arrow[r] &Y
    \end{tikzcd}
  \]
  then we must have a lift $h: \Delta^n \to X$ which commutes with the diagram,
\item a weak equivalence if $|f|: |X|\to |Y|$ is a weak equivalence in
  \topp,
\item a cofibration if it is injective.
\end{itemize}
\end{definition}
\begin{lemma}
  This defines a model structure on $\msset$.
\end{lemma}
\begin{proof}
  (I think we may be proving this later but if not I'm sure it's somewhere
  in \cite{Hov}.)
\end{proof}
You may be wondering why we went to the effort of defining the set $I$ when it
appears we haven't used it in defining a model structure on
$\mathsf{sSet}$. Our reason for including it will become apparent
in the following section, where we define weak factorisation systems.

\section{weak factorisation systems and the small object argument}
Sadly, given an arbitrary model category $\mathcal{M}$, we can't expect the
factorisation of a map (axiom M3) to be functorial. Indeed if you take
(bounded) chain complexs of abelian groups then the factorisation of the
composite of two maps is not necessarily the same as the
composite of the factorisation of the individual maps. This occurs because of
an extension problem. One way to rectify this is to
simply require your factorisations
to be functorial in M3 as in \cite{Hov}. Another way, however is to consider a
nice subclass of model categories called \textit{cofibrantly generated model
  categories}. Finally we give a useful
method for constructing cofibrantly generated model categories from the
small object argument due to Quillen [REFFF].


\subsection{Weak factorisation systems}
Let us first observe some
properties of model categories as we have defined them.
\\
\indent
From now on, given an arbitrary model category $\mathcal{M}$, let us denote
the classes of fibrations by $Fib$, cofibrations by $coFib$ and weak-equivalences
by $\mathcal{W}$. 
\begin{lemma}\label{modelchar1}
      Given a model category $\mathcal{M}$ and a map $f\in \mathcal{M}(A,B)$, then
    \begin{enumerate}
    \item 
      $f$ is a cofibration if and only if $f\boxslash (Fib\cap \mathcal{W})$
    \item 
      $f$ is an acyclic cofibration if and only if
      $f\boxslash Fib$
    \end{enumerate}
\end{lemma}
The obvious mirror statement is true for fibrations,

\begin{lemma}\label{modelchar2}
      Given a model category $\mathcal{M}$ and a map $f\in \mathcal{M}(A,B)$, then
    \begin{enumerate}
    \item 
      $f$ is a fibration if and only if $(coFib\cap \mathcal{W})\boxslash f$
    \item 
      $f$ is an acyclic fibration if and only if
      $coFib \boxslash f$
    \end{enumerate}
  \end{lemma}

  \begin{proof}
    We begin by proving \ref{modelchar1}.1.
    The forward direction follows immediately by definition. For the reverse
    direction suppose we have a map $f:A\to B$ in $\mathcal{M}$ which lifts against
    all acyclic fibrations. Then by (M3) we may factor $f$ as
    \[
      \begin{tikzcd}
        A\arrow[r, tail, "i"'] & C \arrow[r, two heads, "j"', "\sim"] & B
      \end{tikzcd}
    \]
    for some object $C\in \mathcal{M}$. This gives us the following commutative diagram,
    \[
      \begin{tikzcd}
        A\arrow[r, tail, "i"]\arrow[d, "f"'] & C
        \arrow[d, two heads, "j", "\sim" labls] \\
        B \arrow[r, equal] & B
      \end{tikzcd}
    \]
    and since $f$ lifts against all trivial fibrations we obtain a lift $h: B\to C$.
    It's straightforward to see that the following diagram commutes as a result,
    \[
      \begin{tikzcd}
        A\arrow[r, equal]\arrow[d, "f"'] &
        A \arrow[r, equal] \arrow[d,"g", tail] & A \arrow[d,"f"]\\
        B \arrow[r, "h"'] & C \arrow[r, "j"', "\sim"] & B
      \end{tikzcd}
    \]
    Thus $f$ is a retract of $g$ which is a cofibration and therefore, by (M5),
    $f$ is also a cofibration.
    \\
    \\
    \indent
    The proof of the other statements in lemmas \ref{modelchar1} and \ref{modelchar2}
    are essentially identical.
  \end{proof}
  \begin{corollary}
    \phantom{}
    \begin{enumerate}
    \item $coFib$ is closed under pushouts
    \item $coFib\cap \mathcal{W}$ is closed under pushouts
    \item similarly for $Fib$ and $Fib\cap \mathcal{W}$.
    \end{enumerate}
  \end{corollary}
  \begin{proof}
    Suppose we have a pushout square
    \[
      \begin{tikzcd}
        A\arrow[r,"i"] \arrow[d, tail, "g"']
        & C \arrow[d, "f"] \\
        B\arrow[r,"j"'] & D
      \end{tikzcd}
    \]
    We wish to show that $f$ is a cofibration. By the above lemmas it is
    sufficient to show that $f$ has the left lifting property in relation
    to $Fib\cap \mathcal{W}$. Let us consider the following diagram,
    \[
      \begin{tikzcd}
        A\arrow[r,"i"] \arrow[d, tail, "g"'] & C \arrow[d, "f"]\arrow[r] & X
        \arrow[d, two heads, "\sim" labln]\\
        B\arrow[r,"j"'] & D \arrow[r] &Y
      \end{tikzcd}
    \]
    If the righthand square (and hence the whole diagram)
    commutes     then
    since $g$ is a cofibration we get a lift of the outer square $h:B\to X$.
    Now by appealing to the universal property of a pushout square we obtain
    a map $h^\prime : D\to X$ which is a lift for $f$, and hence $f$ is a
    cofibration.
    \\
    \\
    \indent
    The proof for the other cases is similar.
  \end{proof}
  \begin{notation}
     Given a set of maps $S$ in a category
  $\mathcal{D}$ let us denote the set of maps which lift on the left of
  all maps in $S$ by $\prescript{\boxslash}{}{S}$, and all maps which lift on
  the right by $S^{\,\boxslash}$. We say a map $\sigma\in \prescript{\boxslash}{}{S}$
  has the left-lifting property with respect to $S$, and similarly for
  $\sigma \in S^{\,\boxslash}$.
  \end{notation}
  We next present an important definition.
  \begin{definition}
    Given a category $\mathcal{C}$ a weak factorisation system (WFS) on $\mathcal{C}$
    is a pair $(L,R)$, where $L$ and $R$ are classes of maps satisfying the
    following conditions,
    \begin{enumerate}
    \item any map $f\in \mathcal{C}(X,Y)$ factors as $
      \begin{tikzcd}
        X\arrow[r, "i"] & Z \arrow[r, "j"] & Y
      \end{tikzcd}
      $ where $i\in L$ and $j\in R$.
    \item $L=\prescript{\boxslash}{}{R}\,$ and $\,R=L^{\,\boxslash}$
    \end{enumerate}
  \end{definition}

  \begin{example}
    By lemma \ref{modelchar1} and \ref{modelchar2}
    any model category $\mathcal{M}$ has at least
    two weak factorisation systems
    on itself, namely $(coFib, Fib\cap \mathcal{W})\,$ and $\,
    (coFib\cap \mathcal{W}, Fib)$.
  \end{example}
  
  \subsection{Cofibrantly generated model categories}
  
  \begin{definition}
    Given a set $I$ of maps in $\mathcal{C}$, a \textit{relative $I$-cell}
    is the pushout of transfinite-compositions of maps in $I$. We denote
    to the set of $I$-cells in $\mathcal{C}$ by cell$(I)$.
  \end{definition}
  Let us unpack this. First suppose we have some ordinal
  $\alpha$, and some functor $X_\bullet:\alpha\to \mathcal{C}$
  with the properties that

  \begin{enumerate}
  \item the successor map is sent to a` map $X_\bullet(s):X_\bullet \to
    X_{\bullet+1}$ with $X_i(s)\in I$ for all $i\in \alpha$
  \item given any ordinal $a\in \alpha$ we have that:
    $X_a = \colim X_\bullet$ where the colimit is taken over the set
    of ordinals less than $a$.
  \end{enumerate}
  Then a relative $I$-cell is the map $X_0\to X_\alpha := \colim X_\bullet$
  induced by composing the successor maps,
  \[
    \begin{tikzcd}
      X_0\arrow[r,"X_0(s)"] & X_1 \arrow[r,"X_1(s)"] &X_2 \arrow[r]& \cdots
      \arrow[r] &X_\alpha
    \end{tikzcd}
  \]
  It's easy to see 
  that the definition presented in class is this definition in the special
  case that $\alpha= \omega$.

  \begin{definition}
    An object $A\in \mathcal{C}$  is a
    \textit{small object in $\mathcal{C}$} (or a $\kappa$-small object)
    if there exists some cardinal $\kappa$ such that the
    functor $\hom_\mathcal{C}(A,\underline{\phantom{A}}\,) $ preserves $\kappa$-directed colimits.
  \end{definition}
  The last property (which is refered to as being \textit{$\kappa$-compact})
  says that given some poset $\mathsf{J}$ with the property that any
  sub-poset $\mathsf{J}^\prime$ of cardinality $<\kappa$ has an upper-bound in
  $\mathsf{J}$,  and some functor $F: \mathsf{J}\to \mathcal{C}$,
  then the canonical map
    \[
      \begin{tikzcd}
        \colim_{\mathsf{j} \,\in \mathsf{J}} \hom_\mathcal{C}(A,
        F(\mathsf{j}))\arrow[r]
        & \hom_\mathcal{C}(A,
        \colim_ {\mathsf{j} \,\in \mathsf{J}} F(\mathsf{j}))
      \end{tikzcd}
    \]
    arising from the universal property of the lefthand side (after mapping each
    $\hom_\mathcal{C}(A,F(d)) \to \hom_\mathcal{C}(A, \colim_d F(d))$ by
    post-composition with the obvious inclusion $F(d)\to \colim_d F(d)$)
    is actually an isomorphism.

    We are also interested in the case where we have some full-subcategory
    $\mathcal{D}\subset \mathcal{C}$. In this case our
    we simply replace the functor above by $F: \mathsf{J}\to \mathcal{D}$,
    and leave everything else the same. In this context we say
    $A$ is \textit{small relative to $\mathcal{D}$}.


    
  \begin{definition}
    A model category $\mathcal{M}$ is \textit{cofibrantly generated} if
    there are sets of morphisms $I,J$ such that
    \begin{enumerate}
    \item The retracts of any $I$-cell are all the
      cofibrations
    \item The retracts of any $J$-cell are all the
      acyclic cofibrations
    \item The domains of maps in $I$ are small relative to
      $\cell(I)$
    \item The domains of maps in $J$ are small relative to
      $\cell(J)$
    \end{enumerate}
  \end{definition}
  The idea of this definition is that we think of $I$ as generating the set
  of cofibrations, and we think of $J$ as generating the trivial cofibrations.
  We make this precise in the following proposition,

  \begin{proposition}\label{generators}
    Suppose $\mathcal{M}$ is a cofibrantly generated model category.
    Then the retracts of $\cell(I)$ are exactly the elements of
    \,
    $\prescript{\boxslash}{}{\left( I ^{\,\boxslash}\right)}$,
    and the retracts of $\cell(J)$ are exactly the elements of
    \,
    $\prescript{\boxslash}{}{\left( J ^{\,\boxslash}\right)}$
  \end{proposition}
  The proof will be delayed until later.
  \subsection{The small object argument}

  \begin{theorem}[Small Object Argument]
    Let $I$ be a set of maps in a category $\mathcal{C}$.
    If $\mathcal{C}$ is cocomplete, and if the domain of any map in $I$ is
    small, then all maps $f\in \mathcal{C}(X,Y)$ factor as
    \[
      \begin{tikzcd}
        X\arrow[r, "i"] & Z \arrow[r, "s"] & Y
      \end{tikzcd}
    \]
    where $i$ is a relative $I$-cell, and $s\in I^{\,\boxslash}$ 
  \end{theorem}
  \begin{proof}
    Given some morphism $f\in \mathcal{C}(X,Y)$
    consider the set of all commutative squares of the form
    \[
      \begin{tikzcd}
        A\arrow[r]\arrow[d,"i"'] & X \arrow[d,"f"] \\
        B \arrow[r] & Y
      \end{tikzcd}
    \]
    where $i\in I$.  Call this set of squares $S$. Taking coproducts
    gives rise to the following diagram,
    \[
      \begin{tikzcd}
        \coprod_{s\,\in S} A_s \arrow[d, "\coprod i_s"'] \arrow[r] & X
        \arrow[d,"f"]\\
        \coprod_{s\,\in S} B_s \arrow[r] & Y
      \end{tikzcd}
    \]
    We can take a pushout to get the following diagram,
    \[
      \begin{tikzcd}
        \coprod_{s\,\in S} A_s \arrow[d, "\coprod i_s"'] \arrow[r] & X
        \arrow[d] \arrow[ddr,bend left, "f"] &\\
        \coprod_{s\,\in S} B_s \arrow[r]\arrow[drr, bend right] & Z_1
        \arrow[dr, dashed] & \\
        & & Y
      \end{tikzcd}
    \]
    Choose some cardinal $\kappa$ such that all domains in $I$ are
    $\kappa$-small. We may assume that $\kappa$ is infinite since
    if $\alpha<\beta$ are cardinals then $\alpha$-smallness
    implies $\beta$-smallness.
    We continue this inductively over the natural numbers. Given the limit
    ordinal $\omega$, we then define $Z_\omega$ to be
    $\colim_{\alpha<\omega} Z_\alpha$. We repeat this process --
    taking pushout squares for all successor ordinals, and
    taking the colimit of all small successor ordinals
    for limit ordinals. This gives us a map
    $X\to Z$ (which is clearly a relative $I$-cell),
    and hence also the following commutative square,
    \[
      \begin{tikzcd}
        A\arrow[r]\arrow[d] & Z \arrow[d, "s"]\\
        B\arrow[r] & Y
      \end{tikzcd}
    \]
    The final thing for us to show is that $s\,\in I^{\,\boxslash}$.
    To see that we wish to produce a lift $B\to Z$.
    Since we assumed that any possible choice of $A$ is
    $\kappa$-small we know that the map $A\to Z$
    factor through $Z_\alpha$ for some $\alpha<\kappa$.
    Taking the pushout of the obvious diagram gives us a map
    $B\to Z_{\alpha+1}$ which produces the lift we were looking for
    by composing with the inclusion $Z_{\alpha+1}\hookrightarrow Z$. 
    Hence we may
    factor the map $X\to Y$ as $$X\to Z\to Y$$
    where the map $X\to Z$ is a relative $I$-cell, and where the map
    $Z\to Y$ has the right lifting property with respect to
    $I$.
  \end{proof}
  \begin{remark}
    The set-theoretic reason why we ask for $I$ to be a set and not a class
    is because we want to be guarenteed the existence of a single
    cardinal $\kappa$ such that each of the domains of maps in $I$
    is $\kappa$-small. If we dropped this requirement, it is possible that
    we could find some model category, some $I$ (perhaps a proper class),
    and some $f$ such that no
    $\kappa$ exists, and then our argument would fall apart.
  \end{remark}
  We now return to the proof of $\ref{generators}$.
  \begin{proof}[Proof of \ref{generators}]
   
    The proof is the same regardless of choice of $I$ or $J$, so consider $I$.
    Clearly, any retract of an $I$-cell is in
    $\prescript{\boxslash}{}{\left( I ^{\,\boxslash}\right)}$, so we only
    need to show the reverse inclusion.

    Suppose we have an element $f:A\to C\in
    \prescript{\boxslash}{}{\left( I ^{\,\boxslash}\right)}$. Then
    we may factor this (by the small-object argument) as the
    composition of maps 
    \[
      \begin{tikzcd}
        A \arrow[r, "i"] & B \arrow[r, "s"] &C
      \end{tikzcd}
    \]
    where $i$ is a retract of an element of $\cell(I)$, and where
    $s$ is in $I^{\,\boxslash}$. We then get the following commutative diagram,
    \[
      \begin{tikzcd}
        A \arrow[d, "f"'] \arrow[r, "i"]& B \arrow[d, "s"] \\
        C \arrow[r, equals] & C
      \end{tikzcd}
    \]
    and since $s\in I^{\,\boxslash}$, we obtain a lift $\ell :
    C\to B$.
    This fits into the following commutative diagram,
    \[
      \begin{tikzcd}
        A \arrow[d, "f"'] \arrow[r, equals]& A \arrow[d, "i"] \arrow[r,equals]
        & A \arrow[d, "f"]\\
        C \arrow[r, "\ell"'] & B \arrow[r, "s"'] & C
      \end{tikzcd}
    \]
    and thus $f$ is a retract of an element of $\cell(I)$ as we wanted to show.
  \end{proof}
  \subsection{Constructing cofibrantly generated model categories}
  \begin{theorem}
    Suppose you have a bicomplete category $\mathcal{C}$, a subcategory
    $\mathcal{W}\subset \mathcal{C}$, and two sets of maps
    $I$ and $J$. Then these produce a cofibrantly generated model
    structure if and only if
    \begin{enumerate}
    \item $\mathcal{W}$ satisfies the $2/3$ property
    \item The domains of maps in $I$ are small relative to $\cell(I)$
    \item The domains of maps in $J$ are small relative to $\cell(J)$
    \item $\cell(J)\subset \mathcal{W}\cap \prescript{\boxslash}{}{
        (I^{\,\boxslash})}$
    \item $I^{\,\boxslash}\subset \mathcal{W}\cap J^{\,\boxslash}$
    \item Either \,\,$\mathcal{W}\cap \prescript{\boxslash}{}{
        (I^{\,\boxslash})} \subset \prescript{\boxslash}{}{
        (J^{\,\boxslash})}$ \,\,or\,\, 
      $\mathcal{W}\cap J^{\,\boxslash}\subset I^{\,\boxslash}$
    \end{enumerate}
  \end{theorem}
  \begin{proof}
    Clearly if $\mathcal{C}$ is a cofibrantly generated model
    category, then these conditions are satisfied.
    Let us consider the converse situation.

    The axioms M1, M2, and M5 are satisfied trivially by definition.
    The factorisation axiom M3 follows from the small object argument.
    Hence all that's left to check is the lifting axiom M4.

    Let us first assume that 
    $\mathcal{W}\cap \prescript{\boxslash}{}{
        (I^{\,\boxslash})} \subset \prescript{\boxslash}{}{
        (J^{\,\boxslash})}$. Clearly then trivial cofibrations
      lift against fibrations since $J^{\,\boxslash}$ 
  \end{proof}

%%%%%%%%%%%%%%%%%%%%%%%%%%%%%%%%%%
%%%%%%%%%%%%%%%%%%%%%%%%%%%%%%%%%%
%%%%%%%%%%%%%%%%%%%%%%%%%%%%%%%%%%
%%%%%%%%%%%%%%%%%%%%%%%%%%%%%%%%%%
% Bibliography %%%%%%%%%%%%%%%%%%%
%%%%%%%%%%%%%%%%%%%%%%%%%%%%%%%%%%
\begin{thebibliography}{10}
\bibitem[May]{May} Jon Peter May, {\it Simplicial Objects in Algebraic Topology}, University of Chicago Press, 1982
\bibitem[EZ]{EZ} Eilenberg, S., \& Zilber, J. (1953). {\it On Products of Complexes}. American Journal of Mathematics, 75(1), 200-204. doi:10.2307/2372629
\bibitem[BK]{BK} Bousfield, A.K., \& Kan, D.M. (1972). {\it Homotopy Limits, Completions and Localizations}. Springer-Verlag Berlin Heidelberg.
\bibitem[Hat]{Hat} Hatcher
\bibitem[Hov]{Hov} Hovey
\end{thebibliography} 


% \begin{thebibliography}{99}

% \bibitem{adamek} Ad\'{a}mek and Rosicky. {\it Locally Presentable
% and Acessible Categories}. Cambridge University Press, Cambridge,
% 1994.

% \end{thebibliography}

\end{document}
