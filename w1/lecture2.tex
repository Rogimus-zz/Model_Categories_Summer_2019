\documentclass{amsart}
\usepackage{todonotes}
\usepackage{amssymb}
\usepackage{mathabx}

\def\Z{\mathbb{Z}}
\def\C{\mathbb{C}}
\def\F{\mathbb{F}}

%\newcommand{\boxtimes}{\odot}
\newcommand{\stacks}{\Delta}
\newcommand{\toposX}{\mathfrak X}
\newcommand{\etale}{\'{e}tale\,}
\newcommand{\Cech}{\v{C}ech\,}
\newcommand{\bigdot}{\bullet}
\newcommand{\coproduct}{\coprod}
\newcommand{\catname}[1]{{\sffamily\upshape{{#1}}}}
\newcommand{\sset}{\catname{sSet}}
\newcommand{\topp}{\catname{Top}}
\DeclareMathOperator{\Sing}{Sing}
\DeclareMathOperator{\Sub}{\mathcal P}
\DeclareMathOperator{\coeq}{\mathfrak CE}
\DeclareMathOperator{\group}{\mathfrak GPD}
\DeclareMathOperator{\colim}{colim}
\DeclareMathOperator{\Top}{\mathsf{Top}}
\DeclareMathOperator{\Natl}{Natl}
%\DeclareMathOperator{\topp}{Top}
\DeclareMathOperator{\msset}{\mathsf{sSet}}
\DeclareMathOperator{\PTop}{Fun^{\text{pres}}}
\DeclareMathOperator{\holim}{holim}
\DeclareMathOperator{\cosk}{cosk}
\DeclareMathOperator{\hocolim}{hocolim}
\DeclareMathOperator{\bd}{\partial}
\DeclareMathOperator{\U}{\mathcal{U}}
\DeclareMathOperator{\calU}{\mathcal{U}}
\DeclareMathOperator{\calW}{\mathcal{W}}
\DeclareMathOperator{\calE}{\mathcal{E}}
\DeclareMathOperator{\calB}{\mathcal{B}}
\DeclareMathOperator{\calK}{\mathcal{K}}
\DeclareMathOperator{\Sp}{Sp}
\DeclareMathOperator{\Simp}{\bold{CX}}
\DeclareMathOperator{\calF}{\mathcal{F}}
\DeclareMathOperator{\calG}{\mathcal{G}}
\DeclareMathOperator{\Hom}{Hom} \DeclareMathOperator{\bP}{\beta P}
\DeclareMathOperator{\HH}{H} \DeclareMathOperator{\BU}{BU}
\DeclareMathOperator{\id}{id} \DeclareMathOperator{\Fun}{Fun}
\DeclareMathOperator{\calC}{\mathcal{C}}
\DeclareMathOperator{\calI}{\mathcal{I}}
\DeclareMathOperator{\calJ}{\mathcal{J}}
\DeclareMathOperator{\SSet}{\mathcal{S}}
\DeclareMathOperator{\calS}{\mathfrak{S}}
\DeclareMathOperator{\calX}{\mathcal{X}}
\DeclareMathOperator{\red}{\text{hyp}}
\DeclareMathOperator{\calY}{\mathcal{Y}}
\DeclareMathOperator{\op}{op}
\DeclareMathOperator{\calD}{\mathcal{D}}
\DeclareMathOperator{\Ind}{Ind} \DeclareMathOperator{\Acc}{Acc}
\DeclareMathOperator{\Pre}{\PTop}
\DeclareMathOperator{\calP}{\mathcal{P}} \topmargin=0in
\oddsidemargin=0in \evensidemargin=0in \textwidth=6.5in
\textheight=8.5in

\newtheorem{theorem}{Theorem}[section]
\newtheorem{lemma}[theorem]{Lemma}
\newtheorem{proposition}[theorem]{Proposition}
\newtheorem{corollary}[theorem]{Corollary}
\newtheorem{fact}[theorem]{Fact}
% \newtheorem{bigtheorem}{Theorem}
\newtheorem{properties}[theorem]{Properties}

\usepackage{tikz-cd}

\theoremstyle{definition}
\newtheorem{definition}[theorem]{Definition}
\newtheorem{example}[theorem]{Example}
\newtheorem{counterexample}[theorem]{Counterexample}
\newtheorem{conjecture}[theorem]{Conjecture}
\newtheorem{postulate}[theorem]{Postulate}
\newtheorem{remark}[theorem]{Remark}

%\newtheorem{notation}[theorem]{Notation}
%\newtheorem{question}[theorem]{Question}
%\newtheorem{variant}[theorem]{Variant}


\begin{document}

\title{Model Categories}
\author{notes anotated by Roger Murray}
\date{\today}

\maketitle
\section{Notation}
Categories will typically be sans serif (\textit{i.e.} \topp, \sset, etc.) or caligraphric
(\textit{i.e.} $\mathcal{C}$, $\mathcal{S}$, etc.).
If $\calC$ is an arbitrary category then we denote the the set of $\calC$-morphisms from
objects $A,B\in \mathcal{C}$ by $\mathcal{C}(A,B)$.

\section{Model Categories}
\begin{definition}
  A model category is a category $\calC$ with three distinct
  collections of 1-morphisms being fibrations, cofibrations, and
  weak-equivalences which satisfies the following axioms:
  
\end{definition}


%%%%%%%%%%%%%%%%%%%%%%%%%%%%%%%%%%
%%%%%%%%%%%%%%%%%%%%%%%%%%%%%%%%%%
%%%%%%%%%%%%%%%%%%%%%%%%%%%%%%%%%%
%%%%%%%%%%%%%%%%%%%%%%%%%%%%%%%%%%
% Bibliography %%%%%%%%%%%%%%%%%%%
%%%%%%%%%%%%%%%%%%%%%%%%%%%%%%%%%%
\begin{thebibliography}{10}
\bibitem[May]{May} Jon Peter May, {\it Simplicial Objects in Algebraic Topology}, University of Chicago Press, 1982
\bibitem[EZ]{EZ} Eilenberg, S., \& Zilber, J. (1953). {\it On Products of Complexes}. American Journal of Mathematics, 75(1), 200-204. doi:10.2307/2372629
\bibitem[BK]{BK} Bousfield, A.K., \& Kan, D.M. (1972). {\it Homotopy Limits, Completions and Localizations}. Springer-Verlag Berlin Heidelberg.
\end{thebibliography} 


% \begin{thebibliography}{99}

% \bibitem{adamek} Ad\'{a}mek and Rosicky. {\it Locally Presentable
% and Acessible Categories}. Cambridge University Press, Cambridge,
% 1994.

% \end{thebibliography}

\end{document}
